Dynamic network analysis (DNA) is an emergent scientific field that brings together traditional social network analysis (SNA), link analysis (LA), social simulation and multi-agent systems (MAS) within network science and network theory. There are two aspects of this field. The first is the statistical analysis of DNA data. The second is the utilization of simulation to address issues of network dynamics. DNA networks vary from traditional social networks in that they are larger, dynamic, multi-mode, multi-plex networks, and may contain varying levels of uncertainty. The main difference of DNA to SNA is that DNA takes interactions of social features conditioning structure and behavior of networks into account. DNA is tied to temporal analysis but temporal analysis is not necessarily tied to DNA, as changes in networks sometimes result from external factors which are independent of social features found in networks. One of the most notable and earliest of cases in the use of DNA is in Sampson's monastery study, where he took snapshots of the same network from different intervals and observed and analyzed the evolution of the network.[1]
DNA statistical tools are generally optimized for large-scale networks and admit the analysis of multiple networks simultaneously in which, there are multiple types of nodes (multi-node) and multiple types of links (multi-plex). Multi-node multi-plex networks are generally referred to as meta-networks or high-dimensional networks. In contrast, SNA statistical tools focus on single or at most two mode data and facilitate the analysis of only one type of link at a time.
DNA statistical tools tend to provide more measures to the user, because they have measures that use data drawn from multiple networks simultaneously. Latent space models (Sarkar and Moore, 2005)[2] and agent-based simulation are often used to examine dynamic social networks (Carley et al., 2009).[3] From a computer simulation perspective, nodes in DNA are like atoms in quantum theory, nodes can be, though need not be, treated as probabilistic. Whereas nodes in a traditional SNA model are static, nodes in a DNA model have the ability to learn. Properties change over time; nodes can adapt: A company's employees can learn new skills and increase their value to the network; or, capture one terrorist and three more are forced to improvise. Change propagates from one node to the next and so on. DNA adds the element of a network's evolution and considers the circumstances under which change is likely to occur.

An example of a multi-entity, multi-network, dynamic network diagram
There are three main features to dynamic network analysis that distinguish it from standard social network analysis. First, rather than just using social networks, DNA looks at meta-networks. Second, agent-based modeling and other forms of simulations are often used to explore how networks evolve and adapt as well as the impact of interventions on those networks. Third, the links in the network are not binary; in fact, in many cases they represent the probability that there is a link.
Meta-Network A meta-network is a multi-mode, multi-link, multi-level network. Multi-mode means that there are many types of nodes; e.g., nodes people and locations. Multi-link means that there are many types of links; e.g., friendship and advice. Multi-level means that some nodes may be members of other nodes, such as a network composed of people and organizations and one of the links is who is a member of which organization.
While different researchers use different modes, common modes reflect who, what, when, where, why and how. A simple example of a meta-network is the PCANS formulation with people, tasks, and resources.[4] A more detailed formulation considers people, tasks, resources, knowledge, and organizations.[5] The ORA tool was developed to support meta-network analysis.[6]

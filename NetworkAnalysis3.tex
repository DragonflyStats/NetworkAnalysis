\section{Network analysis - cost scheduling}
%----------------------------------%
\subsection{Objectives}
After studying this chapter you will 
\begin{itemize}
\item  understand the principles of least cost scheduling or 'crashing' the network 
\item  know the meaning of: normal and crash costs, normal and crash times, and cost slopes 
\item  be able to use the rules of least cost scheduling 
\item  know how to crash a network. 
\end{itemize}
%----------------------------------%
\subsection{Costs and networks}
\begin{itemize}
\item A further important feature of network analysis is concerned with the costs of activities 
and of the project as a whole. This is sometimes known as PERT/COST. 
\item Cost analysis objectives. The primary objective of network cost analysis is to be able to 
calculate the cost of various project durations. The normal duration of a project incurs a 
given cost and by more labour, working overtime, more equipment etc the duration 
could be reduced but at the expense of higher costs. Some ways of reducing the project 
duration will be cheaper than others and network cost analysis seeks to find the cheapest 
way of reducing the overall duration. 
\item Penalties and Bonuses. A common feature of many projects is a penalty clause for 
delayed completion and/or a bonus for earlier completion. In examination questions, 
network costs analysis is often combined with a penalty and/or bonus situation with the 
general aim of calculating whether it is worthwhile paying extra to reduce the project 
time so as to save a penalty. 
\end{itemize}
%----------------------------------%
\section{Cost and networks - basic definitions }
\begin{itemize}
\item Normal cost. The costs associated with a normal time estimate for an activity. Often 
the 'normal' time estimate is set at the point where resources (men, machines etc) are 
used in the most efficient manner. 
\item Crash cost. The costs associated with the minimum possible time for an activity. 
Crash costs, because of extra wages, overtime premiums, extra facility costs are 
always higher than normal costs. 
\item Crash time. The minimum possible time that an activity is planned to take. The 
minimum time is invariably brought about by the application of extra resources, e.g. 
more labour or machinery. 
\item Cost slope. This is the average cost of shortening an activity by one time unit (day, 
week, month as appropriate). The cost slope is generally assumed to be linear and is 
calculated as follows: 
\[ Cost slope  \frac{Crash cost - Normal cost }{Normal time - Crash time } \]

%- 343 

%======================================%


e.g. Activity A data: 
Normal Crash 
Time I Cost Time I Cost 12 days at £480 8 days at £640 640 - 480 Cost slope = 12 - 8 = f.,401day  

\item Least cost scheduling or 'crashing'. The process which finds the least cost method of reducing the overall project duration, time period by time period. 
The following example shows the process step by step. 



\section{Least cost scheduling rules} 3. The basic rule of least cost scheduling is simply stated. 
Reduce the time of the activity on the critical path with the lowest cost slope and progressively repeat this process until the desired reduction in time is achieved. 
Complications occur when time reductions cause several paths to become critical simultaneously thus necessitating several activities to be reduced at the same time. 
These complications are explained below as they occur. 

Least cost scheduling example A project has five activities and it is required to prepare the least cost schedules for all possible durations from 'normal time' - 'normal cost' to 'crash time' - 'crash cost'. 
Project data activity Preceding activity Time (days) Normal Crash Costs (£) Normal Crash Cost (4) Slope 

A — 4 3 360 420 60 
B — 8 5 300 510 70 
C A 5 3 170 270 50 
D A 9 7 220 300 40 
E B,C 5 3 200 360 80 

Project network. Figure 2411 
Project durations and costs a) Normal Duration 14 days  Critical path A, C, E Project cost (i.e. cost of all activities at normal time) = £1.250  (i,e, £360 + 300 +170+ 220+ 200) 
344 
1 


ist cost method of )(I. The following 
e of the activity on s process until the reductions cause oral activities to be they occur. 

A schedules for all `crash cost'. Cost (£) Slope 
tsh 
60 .0 70 TO 50 40 .)0 80 
24 Network analysis - cost scheduling 
b) Reduce by 1 day the activity on the critical path with the lowest cost slope. Reduce activity C at extra cost of £50 Project Duration 13 days  Project cost £1,300  Note: All activities are now critical. c) Several alternative ways are possible to reduce the project time by a further 1 day but note 2 or 3 activities need to be shortened because there are several critical paths. 
Possibilities available Reduce by day Extra Costs Activities critical 
A and B £60 + 70 = £130 All 
D and E £40 + 80 = £120 All 
B, C and D £70 + 50 + 40 = £160 All 
A and E £60 +80 = £140 A,D,B,E 

An indication of the total extra costs apparently indicates that the second alternative (i.e. D and E reduced) is the cheapest. 
However, closer examination of the last alternative (i.e. A and E reduced) reveals that activity C is non-critical and with 1 day float. 
It will be recalled that Activity C was reduced by 1 day previously at an extra cost of £50. 
If in conjunction with the A and E reduction, Activity C is increased by 1 day, the £50 is saved and all activities become critical. 
The net cost therefore for the 12 day duration is £1,300 + (140 - 50) = £1,390. The network is now as follows: 

\[IAMGE\]

Duration 12 days Cost £1,390 All activities critical 
d) the next reduction would be achieved by reducing D and E at an increase of £120 with once again all activities being critical. Project duration 11 days Project cost £1,510  The final reduction possible is made by reducing B, C and D at an increased cost of £160. The final network becomes: 
10 
Duration 10 days Cost £1,670 All activities critical 
345 



%======================================%
All the principles necessary to crasn networics nave already been covered and the following points may save time in an examination.
\begin{enumerate}[(a)]
\item Only critical activities affect the project duration so take care not to crash non-critical activities. 
\item The minimum possible project duration is not necessarily the most profitable option. It may be cost effective to pay some penalties to avoid higher crash costs. 
\item  If there are several independent critical paths then several activities will need to be crashed simultaneously. 
If there are several critical paths which are not separate i.e. they share an activity or activities, then it may be 
cost effective to crash the shared activities even though they may not have the lowest cost slopes. 
\item  Always look for the possibility of increasing the duration of a previously crashed activity when subsequent crashing renders it non-critical, i.e. it has float. 
\end{enumerate}
%----------------------%
\subsection{Summary} 5. a) Cost analysis of networks seeks the cheapest ways of reducing project times. b) The crash cost is the cost associated with the minimum possible time for an activity, which is known as the crash time. c) The average cost of shortening an activity by one time period (day, weeks etc) is known as the cost slope. d) Least cost scheduling finds the cheapest method of reducing the overall project time by reducing the time of the activity on the critical path with the lowest cost slope. 
Points to note 6. a) The total project cost includes all activity costs not just those on the critical path. 
b) The usual assumption is that the cost slope is linear. This need not be so and care should be taken not to make the 
linearity assumption when circumstances point to some other conclusion. c) The example used in this chapter includes increasing 
the time of a subcritical activity, which has already been crashed, so saving the extra costs incurred. Always look for such possibilities. 
d) Dummy activities have zero slopes and cannot be crashed. 
\subsection{Self review questions} Numbers in brackets refer to paragraph numbers 1 What is the objective of network cost analysis? (2) 2 What are 'normal' costs and 'crash' costs? 
(2) 3 What is the basic rule of least cost scheduling? (3) 
LV LI clblt ItOlt-crincat st profitable option. ;h costs. ties will need to be are not separate i.e. to crash the shared previously crashed as float. mject times. time for an activity, (day, weeks etc) is overall project time mest cost slope. le critical path. not be so and care umstances point to subcritical activity, d. Always look for 
2. a WIC . - activity Normal Crash Normal 1 6 4 500 2 4 2 300 3 1 7 6 650 4 1 3 2 400 5 2,3 5 3 850 Construct a least cost schedule for the network in question 1 showing all normal time - normal cost to crash time - crash cost. Cost Crash 620 390 680 450 1,000 durations from 
\subsection{Answers to exercises}

\begin{enumeratae}
\item 1. Critical path 

Cost slopes 
Activity Cost slope 1 £60 2 45 3 30 4 50 5 75 
\item 2. Total normal cost £3,700 with 17 day duration Cost 
16 day duration (Activity 3) £2,730 
15 day duration (Activity 1) £2,790 
14 day duration (Activity 1) £2,850 
13 day duration (Activity 5) £2,925 
12 day duration (Activity 5) £3,000 
317 

